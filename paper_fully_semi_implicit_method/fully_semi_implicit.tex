\documentclass{article}

\usepackage[margin=1in]{geometry}

\begin{document}

% Title
\title{Comparing Semi-Implicit and Full-Implicit Method for Solveing Stiff Density Dependent Diffusion-Reaction Equations Arising in Biofilm Growth Models}
\author{Hermann J. Eberl and Eric M. Jalbert\\
  (heberl@uoguelph.ca, ejalbert@uoguelph.ca)\\ \\
  Dept. of Mathematics and Statistics, University of Guelph,\\ 
  Guelph, ON N1G 2W1, Canada}
\maketitle

\begin{abstract}
  This is the abstract. The discussion of the difference between fully-implicit and semi-implicit method will be discussed in this paper.
\end{abstract}

\section{Intro}
  Bacterial biofilms are communities of microorganisms that adhere on immersed aquaeous surfaces that can support microbial growth based on the environmental conditions. 
  Biofilms are prevalent on both organic or inorganic surfaces in natural, industrial, or hospital settings.
  When attached to a surface, these bacterias embed themselves in a self-produced extracellular polymeric substance (EPS).
  The EPS provides a layer of protection against washout and biocides, making harmful biofilms difficult to remove.
  This attribute of biofilms is a problem since they are associated with medical issues in the form of, bacterial infections, dental plaque, and other diseases and industrial issues, such as biocorrosion of water pipes. 
  They also contribute positivly to wastewater treatment, groundwater protection, soil remediation, and other environmental engineering technologies.
  
  
  % Describe the history of modelling these systems.
  To further the understanding of biofilms, a diffusion-reaction model for biofilm growth was proposed in \cite{eberl2001deterministic}.
  This model is based 
  
  % Talk about how to solve these systems.
 
   
  % Mention that some methods are semi-implicit.
  % Define semi-implicit (with a reference).
  % Mention that others are fully-implicit.
  % Define fully-implicit.
  % Have a little bit on the difference between semi-implicit methods and fully-implicit (which should be relativly clear from the definitions).
 
  % Talk about the difference in what I'm doing here (i.e. that using a fully-implicit method won't add too much time but still be accurate, or that it doesn't matter which you use since there isn't a difference in accuracy). 

\section{Model}
  % Talk about the biology of the system (alex paper).
 
  % Describe the system mathematically (what I've been doing all along)
  % Talk about what each function represents, and what each equation is. 
  % Also mention the significance of each parameter  if possible.
  % Mention the initial conditions and the region that is solved on (square?)
  % With the region, mention the boundary conditions (Neumann)

\section{Method}
  % Talk about the numerical methods used (trapezidral rule, finite difference method, and Conjugate Gradiant method). 
  % Probably mention that it's stored in a diagonal format.
  % Talk about the grid division and ordering.
  % 
  % Actually list the discretized system, with the grid ordering (report 02)
  % 
  % Find some way to prove that the discretized system is positive definite, symmetric, and diagonally dominent (have a Propsition with proof?)
  % Now with theses characteristics we can sove it using the Conjugate Gradiant method (according the the Y Saad Iterative methods for sparse linear systems).

\section{Results}
  \subsection{Simulation Setup}
    % Mention region, grid size, number of grids (i.e 1024 x 1024 or something).
    % Mentions what language the code is written in
    % " any libraries used.... (none, openMP?  " the platform used (Reckoning2!, altixuv?, Dell-T-1600?)

  \subsection{Results}
    % Mention the parameter values used (or reference the appendix that holds them)
    % Show a few snap shots of the map view?
    % Describe, biologically, what the soltion shows?

  \subsection{Comparisons}
    % Here, report the computation time, max/average num. of iterations for the linear solver AND the fully implicit iteration.
    % Also mention the Accuracy of the solutions by using the norm of each solution compared to the  most accurate one.... (Maybe take a solution at a higher grid resolution?, not sure what to do here)
    % Take differen solutions with different level of accuracry for the fully-implicit iterations (i.e. change eSoln)

\section{Conclusions}
  %Make conclusions....
  


\bibliographystyle{plain}
\bibliography{References}


\end{document}
