%Abstract
\phantomsection
\addcontentsline{toc}{section}{Abstract}
%\section*{ABSTRACT}
\begin{abstract}
\setcounter{page}{2}
\begin{center}
%\vspace*{33pt}
\large{\textbf{Using an ADM-Based Model to Explore Human Intestinal Flora Behaviour}}\\
\vspace*{20pt}
\begin{minipage}{0.49\textwidth}
\begin{flushleft}
\normalsize{Arun S. Moorthy}\\
\normalsize{University of Guelph,2011}
\end{flushleft}
\end{minipage}
\begin{minipage}{0.49\textwidth}
\begin{flushright}
\normalsize{Advisor}\\
\normalsize{Professor Hermann J. Eberl}
\end{flushright}
\end{minipage}
\end{center}
\vspace*{20pt}
\onehalfspacing
The human colon is an anaerobic environment densely populated with bacterial species, creating what is known as the \textit{human intestinal microbiome}; an ecosystem imperative to physiological function with regards to metabolism of non-digestible residues, growth of cells and immune protection from invading organisms. As such, quantifying, and subsequently developing an understanding of the behaviour of this microbial population can be of great value. Unfortunately, because of the physical inaccessibility of many parts of the gastro-intestinal (GI) tract, routine experimentation with this environment is not practical. However, theoretical modelling techniques including \textit{in vitro} and \textit{in silico} simulation/experimental platforms provide a means by which further studying of intestinal microflora can be approached. Perfecting these theoretical models is an important step in further understanding colon microbiota.

An existing \textit{in silico} model of carbohydrate digestion in the colon, developed by Mu\~{n}oz-Tamayo et al. (2010) has been used as a platform for experimentation with the intention of of discovering features which may be removed and/or added to improve the performance and reliability of the design. The model is an adaptation of the waste-water engineering based mathematical model ADM1 (Anaerobic Digestion Model 1), developed to incorporate biochemical and environmental specifications as well as physical structures particular to the human colon. The model is then a system of 102-ordinary differential equation with 66 parameters. 

Simulations with the default model configuration as well as variations of input variables, namely \textit{dietary fiber consumption} and \textit{system flow rate}, were completed to study the effect on \textit{average biomass concentration}, demonstrating significant sensitivity to input variables and an unexpected linearity based on the non-linearity of the original complex system. Simulations and further study suggest that advancements in \textit{in silico} modelling of the colon rely on the development of a metric or scheme that can effectively compare mathematically generated data with that collected through traditional experimentation. Also, experimenting with various \textit{reactor} configurations as a basis for mathematical modelling may prove simpler configurations capable of generating comparable data to more complicated designs which may then also be applicable to existing \textit{in vitro} representations of the colon.

 


\end{abstract}