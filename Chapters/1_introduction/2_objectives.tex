\section{Objectives}


There are three main objectives of this thesis:
\begin{enumerate}
  \item The numerical development, implementation and validation of the proposed fully-implicit method.
    This includes both the theoretical and computational verification of its solutions.
    Theoretically the method must be shown to converge to a single true solution.
    Computationally there are a number of issues that could arise from the implementation: sinks or sources of biomass could be introduced, certain characteristics could be lost, and mass may no longer be conserved.
    These each were investigated for their existance in our implementation.
    Also, a spatial discretization convergence analysis was conducted to verify that the given solutions agree with selection of grid size.
  
  \item The comparison of the fully-implicit method and the semi-implicit method.
    This is for taking inventory of the changes that the newly developed fully-implicit method has against the semi-implicit method for which it is based upon.
    Multiple aspects of the solutions are compared to verify the added computational effort needed.
    A recommendation on the use for this method is established only after the comparison is made.
  
  \item Numerous simulations are conducted to further the understanding of the mechanisms of \textit{C. Thermocellum}.
    The normal behaviour of the system is observed from running a typical simulation.
    Some interesting observations are made from further simulations: travelling wave solution existance and different spatial effect from the choice of intial condition.
\end{enumerate}

