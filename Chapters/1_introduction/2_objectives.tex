\section{Objectives}

There are three main objectives of this thesis:
\begin{enumerate}
  \item Formulate and test a fully-implicit method for a highly degenrate, highly nonlinear coupled PDE-ODE system modelling \textit{C.thermocellum}.
  \item Compare the fully-implicit method of Objective 1 with a previously introduced semi-implicit method, which it generalizes.
    This is done based on the trade-off between improved accuracy of the method and increased computational effort.
  \item Use the numerical methods developed in Object 1 and 2 to simulate \textit{C.thermocellum} biofilm formation on cellulose sheets, with the goal of understanding better the spatio-temporal dynamics of this system.
%!%
%  \item The numerical development, implementation and validation of the proposed fully-implicit method.
%    This includes both the theoretical and computational verification of its solutions.
%    Theoretically the method must be shown to converge to a single true solution.
%    Computationally there are a number of issues that could arise from the implementation: sinks or sources of biomass could be introduced, certain characteristics could be lost, and mass may no longer be conserved.
%    These each were investigated for their existance in our implementation.
%    Also, a spatial discretization convergence analysis was conducted to verify that the given solutions agree with selection of grid size.
%  
%  \item The comparison of the fully-implicit method and the semi-implicit method.
%    This is for taking inventory of the changes that the newly developed fully-implicit method has against the semi-implicit method for which it is based upon.
%    Multiple aspects of the solutions are compared to verify the added computational effort needed.
%    A recommendation on the use for this method is established only after the comparison is made.
%  
%  \item Numerous simulations are conducted to further the understanding of the mechanisms of \textit{C. Thermocellum}.
%    The normal behaviour of the system is observed from running a typical simulation.
%    Some interesting observations are made from further simulations: travelling wave solution existance and different spatial effect from the choice of intial condition.
\end{enumerate}

