\section{Lessons Learned}

\begin{itemize}
%  \item What is the main idea of the lesson? From method formulation/validation
%    Where did we learn this from?
%    How does this help things?
  \item From the numerics chapter of the thesis, the validity and usefulness of a newly developed fully-implicit method was investigated.
    By comparing it to the standard semi-implicit method, which it extends, it was determined that a significant accuracy gain results from a single extra iteration of the fully-implicit method.
    Multiple iterations increase this gain since the increase in accuracy is positively correlated to the number of iterations performed.
    However, the computational effort required from the fully-implicit method grows exponentially with lower tolerance.
    The ratio for solution accuracy when weighed against heavier computation times suggests that two iterations of the fully-implicit method is best (one extra from the semi-implicit method).
    This resulted in an extra digit of accuracy at the cost of approximately $150\%$ the computational effort.
  \item From the simulation chapter of the thesis, a number of useful characteristics were observed in the system.
    The existence of travelling wave solutions was strongly suggested from all the evidence gathered.
    The stability of this wave suggests that it always exists, but this cannot be verified due to the analytic complexity of the problem.
    Testing two sets of initial conditions, chosen at opposing extremes of spatial spreading (all biomass in one location versus equally distributed across one boundary), and measuring the $CO_2$ production showed a large difference between the two solutions at a reactor-scale.
    This suggests that two dimensional models are better for accurately mimicking the behaviour of the system.
\end{itemize}
