\section{Discretization}

In order to approximate the solution for (\ref{equ:model_system}) spatial and temporal discretizations must be made.
First the equations are discretizied in time, 
\begin{equation} \label{equ:M_time_discret}
  \frac{M^{k+1} - M^{k}}{\Delta t} = \nabla_x (D(M^{k+1}) \nabla_x M^{k+1}) + F(C^{k+1}) M^{k+1},
\end{equation}
\begin{equation} \label{equ:C_time_discret}
  \frac{C^{k+1} - C^{k}}{\Delta t} = \frac{1}{2} ( G(C^{k+1}) M^{k+1} + G(C^{k}) M^{k} ).
\end{equation}
Here, (\ref{equ:M_time_discret}) follows the ideas of the Backwards Euler Method; (\ref{equ:C_time_discret}) follows Trapezoidal Rule \citep{burden2010numerical}. 
The index variable $k$ has been introduced in (\ref{equ:M_time_discret}) - (\ref{equ:C_time_discret}) such that $M^{k}(x) \approx M(t^{k}, x)$, allowing an approximation at a certain time, $t^{k}$, to be used; this changes the spatial-temporal continuum model into a spatial continuum model with discrete temporal timesteps. 

For this system, we let the nondimensional spatial region of consideration, $\Omega = [0,1]\times [0,1]$, be square.
Now, only (\ref{equ:M_time_discret}) requires spatial considerations since the substrate does not diffuse across the region.
The spatial discretization will be through the Finite Difference Method as described in \cite{saad2003iterativeMethod}.
Here, a uniform $n \times m$ grid is used to discretize $\Omega$.
Since all the calculations will be done on the grid intersections the discretization will be grid-point based.
This means that a $n \times m$ grid implies there are $(n-1) \times (m-1)$ grid boxes.
The distance between grid points is the same in both $x_1$ and $x_2$ dimensions; we have $\Delta x_1 = \Delta x_2 = \Delta x$.
Since we work on a nondimensionalized domain, and we known the number of grid boxes in our region, we have that $\Delta x = \frac{1}{n-1}$.
A five-point stencil is used to approximate the solution of (\ref{equ:M_time_discret}) at each grid point.
This spatial discritization allows the use of $i$ and $j$ to index across the region such that $x_{1_i} = i * \Delta x$ for $i \in \{ 0, 1, \ldots, n-1 \}$ and $x_{2_j} = j * \Delta x$ for $j \in \{ 0, 1, \ldots, m-1 \}$
To index the grid point, $i$ and $j$ are used such that $M^{k}_{i,j} \approx M(t^{k}, x_{1_i}, x_{2_j})$.
To account for the dependency on neighbouring grid points, we introduce $\sigma$ as the index pair from the set 
\begin{equation}\label{equ:neighbour}
  \mathcal{N}_{ij} = \{n_{ij}, e_{ij}, s_{ij}, w_{ij}\}.
\end{equation}
where, 
\begin{equation}
  \begin{aligned}
    n_{ij} = \begin{cases} 
      (i,j+1)  & \text{ if } j < m \\
      (i,j-1)  & \text{ if } j = m \end{cases}
    & \qquad 
    e_{ij} = \begin{cases}
      (i+1,j)  & \text{ if } i < n \\
      (i-1,j)  & \text{ if } i = n \end{cases}
    \\
    s_{ij} = \begin{cases}
      (i,j-1)  & \text{ if } j > 0 \\
      (i,j+1)  & \text{ if } j = 0 \end{cases}
    & \qquad 
    w_{ij} = \begin{cases}
      (i-1,j)  & \text{ if } i > 0\\
      (i+1,j)  & \text{ if } i = 0 \end{cases}.
  \end{aligned}
\end{equation}
With $\mathcal{N}_{ij}$ and $\sigma$ we can account for the difference in boundary points and interior points.

%!%  I am afraid for somebody who does not know this already it is very difficult to undertsand what you want to say here.  I think the issue is that you give (3.5) without explaining how it is derived or where it comes from.
The equation for (\ref{equ:M_time_discret}), after spatial discretization, is
\begin{equation} \label{equ:M_space_discret_tmp}
  \frac{M^{k+1}_{i,j} - M^{k}_{i,j}}{\Delta t} =
    \frac{1}{\Delta x^2} \sum_{\sigma \in \mathcal{N}_{ij}}
    \left( \frac{D(M^{k+1}_{\sigma}) + D(M^{k+1}_{i,j})}{2} \right)
    \cdot \left( M^{k+1}_{\sigma} - M^{k+1}_{i,j} \right)
    + F(C^{k+1}_{i,j}) M^{k+1}_{i,j}
\end{equation}
For completeness, we spatially discretize (\ref{equ:C_time_discret}) as
\begin{equation} \label{equ:C_space_discret_tmp}
  \frac{C^{k+1}_{i,j} - C^{k}_{i,j}}{\Delta t} = \frac{1}{2} ( G(C^{k+1}_{i,j}) M^{k+1}_{i,j} + G(C^{k}_{i,j}) M^{k}_{i,j} ).
\end{equation}

Notice that for (\ref{equ:M_space_discret_tmp}), the arithmetic mean of the diffusion function, $D$, is taken because of the steep gradient at the interface.
The alternative would be to use $D(M^{k+1}_{i+\frac{s}{2}, j +\frac{r}{2}})$, however in some cases we have $M^{k+1}_{i+\frac{s}{2}, j +\frac{r}{2}} = 0$ which would result in $D(0) = 0$ and thus nullify the effect of the spatial diffusion.
Taking the arithmetic mean eliminates this result because the average value of $D$ would not be zero at the interface.

%!%\begin{figure}
%!% This is just going to be a tikz pic that shows how the values of M may fall between two i indexs so that using i+0.5 => M=0 => D(0)=0 and no diffusion exists VS. using arithmetic mean and M_i+M)i+1 => M = 0.5?? => D(0.5) != 0
%!%\end{figure}

To simplify the spatial indexing, the matrix system is converted into a vector by use of a bijective mapping defined as:
\begin{equation}
\begin{array}{c c c c}
  %!% Make sure this actually maps to {1, \ldots, nm} instead of 1.... (n+1)(m+1)
  \pi :& \{ 0, \ldots, n\} \times \{0, \ldots, m\} & \to & \{0, \ldots, nm \} \\
       & (i,j)                                     & \to & \pi(i,j)
\end{array}
\end{equation}
%!% I have a figure for this?!?!?!?
Now, a single index can be used to iterate over the vector, $l$. 
This gives the system,
\begin{equation} \label{equ:M_space_discret}
  \frac{M^{k+1}_{l} - M^{k}_{l}}{\Delta t} =
    \frac{1}{\Delta x^2} \sum_{\sigma \in \mathcal{N}_{ij}}
    \left( \frac{D(M^{k+1}_{\sigma}) + D(M^{k+1}_{l})}{2} \right)
    \cdot \left( M^{k+1}_{\sigma} - M^{k+1}_{l} \right)
    + F(C^{k+1}_{l}) M^{k+1}_{l}
\end{equation}
\begin{equation} \label{equ:C_space_discret}
  \frac{C^{k+1}_{l} - C^{k}_{l}}{\Delta t} = \frac{1}{2} ( G(C^{k+1}_{l}) M^{k+1}_{l} + G(C^{k}_{l}) M^{k}_{l} ).
\end{equation}

