\section{Discretization}

In order to find the solution for (\ref{equ:model_system}) spatial and temporal discretizations must be made.
First the equations are discretzied in time, 
\begin{equation} \label{equ:M_time_discret}
  \frac{M^{k+1} - M^{k}}{\Delta t} = \nabla_x (D(M^{k+1}) \nabla_x M^{k+1}) + F(C^{k+1}) M^{k+1},
\end{equation}
\begin{equation} \label{equ:C_time_discret}
  \frac{C^{k+1} - C^{k}}{\Delta t} = \frac{h}{2} ( G(C^{k+1}) M^{k+1} + G(C^{k}) M^{k} ).
\end{equation}
Here, (\ref{equ:M_time_discret}) follows the ideas of the Backwards Euler Method; (\ref{equ:C_time_discret}) follows Trapezoidal Rule. 
The index variable $k$ has also been introduced in (\ref{equ:M_time_discret} - \ref{equ:C_time_discret}) such that $M^{k}(x) \approx M(t^{k}, x)$, allowing an approximation at a certain time, $t^{k}$, to be used; this reduces the dimensionality of the problem. 

For this system, the region of consideration will be a rectangular region, $\Omega$.
This region has Neumann boundary conditions, $\frac{\partial M}{\partial x} = \frac{\partial C}{\partial x} = 0, \forall x \in \partial \Omega$.
Now, only (\ref{equ:M_time_discret}) requires spatial considerations since, according to the biology of our system, the substrate does not diffuse across the region.
The spatial discretization will be through the Finite Difference Method as described in \cite{saad2003iterativeMethod}.
Here, a uniform $n \times m$ grid is used to discretize $\Omega$.
This means that the distance between grid points are the same in both $x_1$ and $x_2$ dimensions; we have $\Delta x_1 = \Delta x_2$.
The solution of (\ref{equ:M_time_discret}) will be approximated at each grid point using a five-point stencil. 
To index the grid point, $i$ and $j$ are used such that $M^{k}_{i,j} \approx M(t^{k}, x_{1_i}, x_{2_j})$.
Because of the five-point stencil, the boundary gridpoints will depend on ghost grid points.
This means that the equation to solve interior grid points will differ slightly from boundary points.
The resulting equation for interior points, where $i \in (1, 2, \ldots, n-1)$ and $j \in (1, 2, \ldots, m-1)$, is
\begin{equation} \label{equ:M_space_discret}
  \frac{M^{k+1}_{i,j} - M^{k}_{i,j}}{\Delta t} = 
    \frac{1}{\Delta x^2} \sum_{(s,r) \in \mathbb{A}} 
    \left( \frac{D(M^{k+1}_{i+s,j+r}) + D(M^{k+1}_{i,j})}{2} 
    \cdot ( M^{k+1}_{i+s, j+r} - M^{k+1}_{i,j}) \right) 
    + F(C^{k+1}_{i,j}) M^{k+1}_{i,j}
\end{equation}
where $\mathbb{A} = \left\{ (0, \pm1), (\pm1, 0) \right\}$.
The resulting equation for boundary points, when $(i,j) \in \{0,n\} \times \{0,1,\ldots,m\}$ or $(i,j) \in \{0,1,\ldots,n\} \times \{0,m\}$, is
\begin{equation} \label{equ:M_space_discret_boundary}
  \begin{aligned}
  \frac{M^{k+1}_{i,j} - M^{k}_{i,j}}{\Delta t} =& 
    \frac{1}{\Delta x^2} \sum_{(s,r) \in \mathbb{A}}
    \left( \frac{D(M^{k+1}_{i+s,j+r}) + D(M^{k+1}_{i,j})}{2}
    \cdot ( M^{k+1}_{i+s, j+r} - M^{k+1}_{i,j}) \right) \\&
    + \sum_{(s,r) \in \mathbb{B}}  
    \left( \frac{D(M^{k+1}_{i+s,j+r}) + D(M^{k+1}_{i,j})}{2}
    \cdot ( M^{k+1}_{i-s, j-r} - M^{k+1}_{i+s, j+r}) \right)
    + F(C^{k+1}_{i,j}) M^{k+1}_{i,j}
  \end{aligned}
\end{equation}
where $\mathbb{B}$ depends on the boundary position;
$\mathbb{B}_1 = \left\{ (0,-1) \right\}$, $\mathbb{B}_2 = \left\{ (-1,0) \right\}$, $\mathbb{B}_3 = \left\{ (1,0) \right\}$, $\mathbb{B}_4 = \left\{ (0,1) \right\}$.
For the corner points, where two different boundaries connect, the result is to use a cross-product between $\mathbb{B}$'s, for example $\mathbb{B} = \mathbb{B}_1 \times \mathbb{B}_2$
%!% Maybe add a diagrame (tikz) here that shows where the different B's would lay in the region. 
%!% something like B4-...-B5-...-B6
%!%                |      |      |
%!%                B1-...-B2-...-B3
For both (\ref{equ:M_space_discret} - \ref{equ:M_space_discret_boundary}) the arithmetic mean of the diffusion function, $D$, is taken because of its steep gradiant at the interface.


