\section{Discretization}

In order to find the solution for (\ref{equ:model_system}) spatial and temporal discretizations must be made.
First the equations are discretizied in time, 
\begin{equation} \label{equ:M_time_discret}
  \frac{M^{k+1} - M^{k}}{\Delta t} = \nabla_x (D(M^{k+1}) \nabla_x M^{k+1}) + F(C^{k+1}) M^{k+1},
\end{equation}
\begin{equation} \label{equ:C_time_discret}
  \frac{C^{k+1} - C^{k}}{\Delta t} = \frac{h}{2} ( G(C^{k+1}) M^{k+1} + G(C^{k}) M^{k} ).
\end{equation}
Here, (\ref{equ:M_time_discret}) follows the ideas of the Backwards Euler Method; (\ref{equ:C_time_discret}) follows Trapezoidal Rule. 
The index variable $k$ has also been introduced in (\ref{equ:M_time_discret} - \ref{equ:C_time_discret}) such that $M^{k}(x) \approx M(t^{k}, x)$, allowing an approximation at a certain time, $t^{k}$, to be used; this reduces the dimensionality of the problem. 

For this system, the region of consideration will be a rectangular region, $\Omega$.
This region has Neumann boundary conditions, $\frac{\partial M}{\partial x} = \frac{\partial C}{\partial x} = 0, \forall x \in \partial \Omega$.
Now, only (\ref{equ:M_time_discret}) requires spatial considerations since, according to the biology of our system, the substrate does not diffuse across the region.
The spatial discretization will be through the Finite Difference Method as described in \cite{saad2003iterativeMethod}.
Here, a uniform $n \times m$ grid is used to discretize $\Omega$.
Since all the calculations will be done on the grid intersections the discretization will be grid-point based.
This means that a $n \times m$ grid implies there are $(n-1) \times (m-1)$ grid boxes.
The distance between grid points is the same in both $x_1$ and $x_2$ dimensions; we have $\Delta x_1 = \Delta x_2$.
A five-point stencil is used to approximate the solution of (\ref{equ:M_time_discret}) at each grid point.
To index the grid point, $i$ and $j$ are used such that $M^{k}_{i,j} \approx M(t^{k}, x_{1_i}, x_{2_j})$.
To account for the dependency on neighbouring grid points, we introduce $\sigma$ as the index pair from the set 
\begin{equation}\label{equ:neighbour}
  \mathcal{N}_{ij} = \{n_{ij}, e_{ij}, s_{ij}, w_{ij}\}.
\end{equation}
where, 
\begin{equation}
  \begin{aligned}
    n_{ij} = \begin{cases} 
      (i,j+1)  & \text{ if } j < m \\
      (i,j-1)  & \text{ if } j = m \end{cases}
    & \qquad 
    e_{ij} = \begin{cases}
      (i+1,j)  & \text{ if } i < n \\
      (i-1,j)  & \text{ if } i = n \end{cases}
    \\
    s_{ij} = \begin{cases}
      (i,j-1)  & \text{ if } j > 0 \\
      (i,j+1)  & \text{ if } j = 0 \end{cases}
    & \qquad 
    w_{ij} = \begin{cases}
      (i-1,j)  & \text{ if } i > 0\\
      (i+1,j)  & \text{ if } i = 0 \end{cases}.
  \end{aligned}
\end{equation}
With $\mathcal{N}_{ij}$ and $\sigma$ we can account for the difference in boundary points and interior points.

The equation for (\ref{equ:M_time_discret}), after spatial discretization, is
\begin{equation} \label{equ:M_space_discret}
  \frac{M^{k+1}_{i,j} - M^{k}_{i,j}}{\Delta t} =
    \frac{1}{\Delta x^2} \sum_{\sigma \in \mathcal{N}_{ij}}
    \left( \frac{D(M^{k+1}_{\sigma}) + D(M^{k+1}_{i,j})}{2} \right)
    \cdot \left( M^{k+1}_{\sigma} - M^{k+1}_{i,j} \right)
    + F(C^{k+1}_{i,j}) M^{k+1}_{i,j}
\end{equation}

For (\ref{equ:M_space_discret}), the arithmetic mean of the diffusion function, $D$, is taken because of the steep gradiant at the interface.
The alternative would be to use $D(M^{k+1}_{i+\frac{s}{2}, j +\frac{r}{2}})$, however this may result in a value of zero and thus nullify the effect of the spatial diffusion.


