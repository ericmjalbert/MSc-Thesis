\section{Comparison of Semi-implicit and Fully-implicit Method}
  Here the main comparison that analyses the effects of using Algorithm \ref{alg:iterateCM} with different tolerances. 
  Recall that the main observation is for $tol. = 1$, which correlates to the semi-implicit method since it will allow only a single iteration of the algorithm. 

  The simulation used is the same as described in (\ref{fig:basic_trav}).
  The comparison will be on multiple metrics: the average number of iterations of Algorithm \ref{alg:iterateCM}, the value of $\epsilon_1$ and $\epsilon_2$, the computation time of the simulation, and the location of the wave peak.

  The average number of iterations are tracked so that an idea of the extra work can be formed.
  As the tolerance decreases the amount of iterations the algorithm must perform will increase, the degree of increase will help relate the amount of work.

  The value of $\epsilon_1$ and $\epsilon_2$ act as a measure of accuracy.
  Here, these values correspond to the difference between a pair of solutions, $u_1$ and $u_2$.
  The $(u_1, u_2)$ pairs are: $(1, 10^{-8}), (10^{-8}, 10^{-10}), (10^{-10}, 10^{-12}), (10^{-12}, 10^{14})$.
  Each row of Table \ref{tab:tolerance_comparison} refers to the $u_1$ values of the pairs.
  Each difference was taken at the last timestep.

  Along with accuracy, the simulation time is tracked.
  This is because it represents another metric for which the viability of the fully-implicit method can be verified.
  Theoretically there should be a decrease in the error with the fully-implicit method as the value for $tol$ decreases.
  Therefore, this needs to be weighted against the cost of computational intensity and the increase of the simulation time.

  The location of the wave peak is a tracked quality of the solution that reveals how consistent the results are.
  The wave peak is described here as the maximum value of the solution at the final timestep calculated.
  The ultimate goal is that the simulation solutions be converging towards the exact solution.
  To see this here the $x$-coordinate of the wave peak is tracked.

  The results of the method comparison can be seen in Table \ref{tab:tolerance_comparison}.
  \begin{table}[h!tb]
  \centering
  \begin{tabular}{|c|c|c|c|c|c|}
    \hline
    Tol. & Avg. Iter. & $\epsilon_1$ & $\epsilon_2$ & Time & Wave Peak \\
    \hline
    1.0e-0 & 1.00000000& 0 & 0 & 12.183000000 & 0.46484375 \\
    i.0e-1 & 1.00000000& 0 & 0 & 12.208000000 & 0.46484375 \\
    1.0e-2 & 1.00000000& 0 & 0 & 12.337999999 & 0.46484375 \\
    1.0e-3 & 1.00000000& 0 & 0 & 12.231000000 & 0.46484375 \\
    1.0e-4 & 1.00000000& 0.00200018013 & 0.000754518705 & 12.320000000 & 0.46484375\\
    1.0e-5 & 1.96505087& 2.44746455e-07 & 1.62832139e-07 & 18.986999998 & 0.4609375 \\
    1.0e-6 & 2.00000000& 2.81448171e-07 & 1.94903038e-07 & 19.091000001 & 0.4609375 \\
    1.0e-7 & 2.00187495& 1.12595285e-05 & 7.72252591e-06 & 19.094000001 & 0.4609375 \\
    1.0e-8 & 2.58568535& 7.16665603e-07 & 1.74044990e-07 & 20.516999999 & 0.4609375 \\
    1.0e-9 & 2.90125246& 1.05877879e-05 & 7.00561295e-06 & 21.308000000 & 0.4609375 \\
    1.0e-10 & 3.2278443 & 0.000143911078 & 9.66349067e-05 & 22.228000002 & 0.4609375 \\
    1.0e-11 & 16.099022 & 4.02028944e-05 & 2.71613115e-05 & 57.558999997 & 0.4609375 \\
    1.0e-12 & 36.318492 & 4.23049177e-06 & 2.86249772e-06 & 113.99400000 & 0.4609375 \\
    1.0e-13 & 57.681232 &  &  & 173.84899999999999 & 0.460937500000 \\
    \hline
  \end{tabular}
  \caption{Results from running simulations with different Tol.}
  \label{tab:tolerance_comparison}
\end{table}


  %!% Talk about what is seen here?

  %!% The fully-implicit method should have less iterations of the linear solver.
  
  %\begin{table}[!htp]
  %  \centering
  %  %!% If I just list time and epsilons, maybe four gnuplots graph with the simulation time listed in the legend would be better.
  %  \begin{tabular}{|c|c|c|c|c|}
  %    \hline
  %    P & Simulation Time & $\epsilon_{\ell_1}$ & $\epsilon_{\ell_2}$ & $\epsilon_{\ell_\infty}$  \\
  %    \hline
  %    1& 60.05 & 0.0018 & - & - \\
  %    2& 97.72  & 0.0014 & - & - \\
  %    3& 117.46 & 0.0013 & - & - \\
  %    4& 126.44 & 0.0011& - & - \\
  %    \hline
  %    %!% OLD ROWS - Tol. 1 & Computation Time & $\epsilon_{sol}$ & Avg. Iter. 1 & Max Iter. 1 & Avg. Iter. 2 & Max Iter. 2 \\
  %  \end{tabular}
  %  \caption{Results from running simulation with different $P$ values. Note, $P = 1$ corresponds to the semi-implicit method.}
  %  \label{tab:tolerance_comparison}
  %\end{table}
  
