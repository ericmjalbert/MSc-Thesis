\section{History}

%!% Try to poise the problem as an investigation of whether semi-implicit vs. fully-implicit comparison exists and assuming the fully-iomplici method is more accurate, if we actually gain anything from the extra computations

The tradition biofilm model has been continually developed over many iterations since 1980.
\cite{rittmann1980model} formulated the steady-state biofilm model, developed using the concept that biofilm growth would be the result of a steady flux from substrate.
Since then the model have evolved to include modelling three-dimensional growth of multispecies anaerobic biofilms (\cite{noguera1999simulation}) and spatially heterogeneous biofilm structures (\cite{eberl2001deterministic}). 

The modelling of \textit{Clostridium Thermocellum} is unique because this celluloytic anaerobic bacteria does not generate an extracellular polymeric substance.
This uncharacteristic behaviour means that the mathematical model based on the work of \cite{eberl2007finite} cannot be used as is. 
They modelled the biomass density and nutrient concentrations as a two-PDE-coupled system.
Recently, \cite{wang2011spatial}, used a cellular automata based model for simulating the growth of \textit{Clostridium Thermocellum}. From this, better results were thought to derive from a continuous differential equation based model.
Here the spatial diffusion of the substrait concentration is removed to mimic the carbon substrait that is consumed by \textit{Clostridium Thermocellum}. This results in a PDE-ODE-coupled system.
This is based on the work done by \cite{dumitrache2014understanding}, where this same coupling was used and formulated.
