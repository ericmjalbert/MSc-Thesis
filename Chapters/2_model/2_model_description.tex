\section{Model Description}
The model used for simulations is based on the deterministic biofilm model developed in \cite{eberl2001deterministic}, which was designed for modelling the development of spatially heterogenous biofilm structures.
Since \textit{C.Thermocellum} grows as a monolayered biofilm and consumes a solid carbon fiberous substrate, there are mechanical differences between the two systems.
Our model is based on the following assumptions:
\begin{enumerate}
  \item The growth of sessile biomass is limited locally by the availability of nutrients and by the availability of colonizable space.
    \label{assump:growth_inhib}
  \item The number of cells per unit area of substratum is limited to a finite value because \textit{C.Thermocellum} forms only a thin monolayer. 
    \label{assump:finite_density}
  \item Biomass does not spread until its density approaches the physical limit
    Near the physical limit it expands spatially into neighbouring regions.
    Because of this, the physical limit of biomass density is never attained.
    \label{assump:diffusion}
  \item The carbon fiberous substrate consumed as nutrition for biomass growth is the substratum to which the biofilm attachs.
    Carbon is bound in the fibers of the substratum and does not diffuse.
    The propagation of biofilm is based on the lack of available substrate locally, not the physical degradation of the substratum.
    \label{assump:substratum}
%  \item The model is a volume filling problem. The amount of biomass in a unit of space has a finite limit. Maybe???
%  \item Since \textit{C.Thermocellum} behaves as a monolayer, the area of substrate covered by sessile biomass determines the maximum production of biomass. Further growth of sessile biomass results in greater coverage.
%  \item Since \textit{C.Thermocellum} behaves as a monolayer, the growth of biomass is directly propotional to the area of substrate covered by sessil biomass. 
%  \item The substrate is sessile.
  \item Cell death and cell loss into the aqueous environment is assumed to be proprotional to cell density.
    \label{assump:bio_death}
  \item Biomass growth is proprotional to substrate consumption.
    \label{assump:sub_consumption}
\end{enumerate}

Assumption \ref{assump:growth_inhib}, \ref{assump:finite_density}, \ref{assump:diffusion}, \ref{assump:bio_death}, and \ref{assump:sub_consumption} are similar to those made in \cite{eberl2001deterministic}.
The main difference here is \ref{assump:substratum}; our substrate is sessile.
With a sessile substrate, there is no diffusion for the substrate concentration.
Another difference is that \textit{C. Thermocellum} does not grow from the substratum into the aqueous phase.
Instead our biofilm grows across the substratum making this a two dimensional setting.

The model is formulated in a spatial domain $\Omega$. 
The independent variables $t > 0$ denote time and $x \in \Omega$ denotes the location within the physical domain.
The depependent variables are the local fraction of the surface occupied by biomass $M(t,x)$ and the substrate density $C(t,x)$.
The net growth rate of biomass, in dependence of available substrate we denote by $f(C)$, the substrate consumption rate by $g(C)$.
The diffusion coefficient that describes spatial expansion of biomass is given by the function $d(M)$.

From the above assumptions, a PDE-ODE-coupled system that models \textit{C. Thermocellum} growth on carbonous fibres can be formulated as,
\begin{align} 
   M_t &= \nabla_x \left( d(M) \nabla_x M \right) + f(C) M \label{equ:model_M}\\
   C_t &= -g(C) M \label{equ:model_C}
\end{align}
where
\begin{equation} \label{equ:model_d}
  d(M) = d \frac{M^\alpha}{(1-M)^\beta}
\end{equation}
\begin{equation} \label{equ:model_f}
  f(C) = u \frac{C}{k + C} - n 
\end{equation}
\begin{equation} \label{equ:model_g}
  g(C) = y \frac{C}{k + C}
\end{equation}
with all parameters non-negative.
Here we have a pair of equations, (\ref{equ:model_M}) and (\ref{equ:model_C}), that represent the biomass density and substrate concentration respectivly.
This is a model for the spatial spreading of biomass.
For $0 < M << 1$ the spreading effect is negligible but when $0 << M \approx 1$ we have considerable spreading.
The work done by \cite{khassehkhan2009nonlinearMaster} shows this to be an ideal choice for modelling the spreading of biofilms.
By assumption \ref{assump:growth_inhib}, the only factors effecting the biomass density is growth from nutrient convertion and diffusion from local spatially-full colonized space.
For the density-dependent diffusion equation (\ref{equ:model_d}), $\delta$ is the diffusion coefficient, which controls the magnitude of this term, and the parameters $\alpha$ and $\beta$ are selected to control the strength of the diffusion.
This agrees with assumption \ref{assump:diffusion} since (\ref{equ:model_d}) has a near-zero value until when $M \to 1$, which leads to $d(M) \to \infty$ as seen in Figure \ref{fig:show_d}. 
The production rate is the difference between simple Monod kinetic growth term, with growth rate $u$, and a constant death rate term, $n$, to agree with assumption \ref{assump:growth_inhib} and \ref{assump:bio_death}.
Monod kinetic growth was selected, with half-saturation carbon concentration $k$, since it matches the growth of bacteria when limited by available nutrients. %!% citation here?

Equation (\ref{equ:model_C}) describdes the consumption of cabon substrate due to biomass growth.
Parameter $y$ is the maxmimum consumption rate, measured in mass carbon per unit time.
Substrate consumption is proportional to the local biomass density $M$. 
Parameter $k$, same as in the growth term for (\ref{equ:model_M}) is again the half-saturation carbon concentration.
Here assumption \ref{assump:substratum} and \ref{assump:sub_consumption} are satisified since there exist no diffusion term for the substrate and its growth is a scalar multiple of the biomass growth rate.


\begin{figure}
  \centering
  \begin{tikzpicture}[scale = 4]
    \draw[<->, thick] (1.1,0) -- (0,0) -- (0,1);
    \draw[dashed] (0.95,0) -- (0.95,1.4);

    \node[below] at (0.95,0) {$1$};
    \draw (0.95, 0.02) -- (0.95, -0.02);

    \node[right] at (1.1,0) {$M$};
    \node[above] at (0,1.1) {$d(M)$};

    \draw[->, domain=0:0.92] plot (\x, {0.0001*pow(\x/(1-\x+0.01), 4)});
  \end{tikzpicture}
  \caption{A graph of $d(M) = d \frac{M^\alpha}{(1 - M)^\beta}$ showing the way diffusion increases asymptotically as $M \to 1$.}
  \label{fig:show_d}
\end{figure}
 
The dimensions of the parameters and variables are in Table \ref{tab:varDimensions}.
Note that since we have a two dimensional problem, due to the lack of complex biofilm structures from \textit{C. Thermocellum} growth, the spatial considerations are all strictly for area and not volume, as is typically done for biofilm modelling.
 %!% Also:  I wonder, since we have an entirely 2D problem, whether it would make sense to use specific concentrations/densitites throughout, i.e. mass/area, rather than mass/volume; I think I would prefer that; shouldn't change anything but here in the table, and maye text in section 2.1. DONE! :=> is this correct though?
  \begin{table}[!hbt]
    \centering
    \begin{tabular}{|l |c |l |}
      \hline 
      Description & Symbol & Dimensions \\
      \hline
      \hline
      Spatial region & $\Omega$ &  $NA$ \\
      \hline 
      Time & $t$ & $\left[days\right]$ \\
      Location in $\Omega$ & $x$ & $\left[meters\right]$ \\
      \hline
      Biomass fraction & $M$ & $\left[-\right]$ \\
      Substrate concentration & $C$ & $\left[\frac{grams}{meters^2}\right]$ \\
      \hline
      Diffusion coefficient & $d$ & $\left[\frac{meters}{days}\right]$ \\
      Density-dependent exponent & $ \alpha $ & $\left[-\right]$  \\
      Density-dependent exponent & $ \beta  $ & $\left[-\right]$  \\
      Growth rate & $ u $ & $\left[days^{-1} \right]$ \\
      Half-saturation carbon concentration & $ k $ & $\left[\frac{grams}{meters^2}\right]$ \\
      Maximum consumption rate & $y$ & $\left[\frac{grams carbon}{days}\right]$  \\
      Death constant & $n$ & $\left[\frac{grams}{meters^2 \cdot days}\right]$ \\
      \hline
    \end{tabular}
    \caption{List of parameters and their dimensions}
        \label{tab:varDimensions}
  \end{table}

The model (\ref{equ:model_M}), (\ref{equ:model_C}) is completed by boundary conditions for the biomass density, $M$, and initial conditions for both $M$ and substrate concentration $C$.
For $M$ we pose homogenouse Neumann boundary conditions such that,
\begin{equation} 
  \partial_n M = 0, \quad x \in \partial \Omega.
\end{equation}
The initial conditions for the biomass density is,
\begin{equation}
  M(0,x) = M_0(x), \quad x \in \Omega,
\end{equation}
where $0 \le M_0(x) < 1$ and $M_0(x)$ non-zero in specific pockets on the substratum. 
These are specified below for each indivdual simulation experiments.
The initial conditions for the substrate concentration is,
\begin{equation}
  C(0,x) = C_{\infty}, \quad x \in \Omega,
\end{equation}
where $C_{\infty}$ describes the initial carbon density in the substratum.

There has been shown to exist a finite speed of interface propagation for the solutions of these kinds of degenerate problems, where $d(0) = 0$ and $\alpha >1$ from (\ref{equ:model_d}) \citep{jalbert2014numerical}.
These problems have a blow up in the biomass gradient at the interface because of the degeneracy that exists there.
For this system, we have $M < 1$ always since the diffusion when $M \approx 1$ is great enough to always ensure this.

