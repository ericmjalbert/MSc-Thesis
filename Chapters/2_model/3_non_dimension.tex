\section{Nondimensionalization}
 

To help facilitate the analysis of this system, the full removal of all physical units is preferred and so we nondimensionalize the parameters.
Here the parameters used are: the biomass growth rate, $u$; the length of the region, $L$; and the maximum density for biomass and substrate, $M_\infty$ and $C_\infty$.
The biomass density fraction represents the current density of biomass divided by the maximum biomass density, $M_{\infty}$.
From using the following parameter changes, the system can be made unitless.
\begin{align}
  \chi &= \frac{x}{L} \implies L d\chi= dx \\
  \tau &= u t \implies \frac{1}{u} d\tau= dt \\
  \mathcal{C} &= \frac{C}{C_{\infty}} \\
  \delta &= \frac{1}{u L^2} d \\
  \kappa &= \frac{k}{C_\infty} \\
  \nu &= \frac{n}{u C_\infty} \\
  \gamma &= \frac{M_\infty}{C_\infty} y
\end{align}

Using these, (\ref{equ:model_M}) and (\ref{equ:model_C}) can be simplified and nondimensionalized into, 
\begin{align} \label{equ:model_system}
  M_\tau &= \nabla_\chi \left( D(M) \nabla_\chi M \right) + F(\mathcal{C}) M \\
  \mathcal{C}_\tau &= - G(\mathcal{C}) M, 
\end{align}

where,

\begin{equation}
\begin{aligned} \label{equ:model_functions}
  D(M) &= \delta \frac{M^\alpha}{(1 - M)^\beta} \\
  F(\mathcal{C}) &= \frac{ \mathcal{C}} {\kappa + \mathcal{C}} - \nu \\
  G(\mathcal{C}) &= \gamma \frac{\mathcal{C}}{\kappa +\mathcal{C}}.
\end{aligned}
\end{equation}
with only $\delta, \kappa, \nu, \gamma$ as model parameters. 
For convenience, we henceforth use
\begin{equation}
  C := \mathcal{C},\quad x := \chi,\quad t := \tau.
\end{equation}

Each of the dimensionless parameters in (\ref{equ:model_functions}) have a biological representation based on the transformations done.
The parameter $\delta$ is the dimensionless biomass motility coefficient.
It affects the change in biomass from adjacent biomass sources, a greater $\delta$ results in faster biofilm expansion.
The parameter $\kappa$ is the half-saturation point, it is exactly the value for which substrate concentration results in $0.5$-optimum growth rate.
Parameter $\nu$ is the decay and loss rate for biomass.
These can be from starvation in cases where substrates are depleted or from loss into the aqueous environment.
Lastly, $\gamma$ is the dimensionless maximum substrate consumption rate.
It signifies the ratio of substrate consumed to biomass growth.
Here, a larger $\gamma$ value results in more substrate being consumed to  produce the same amount of biomass. 

With (\ref{equ:model_system}) being reduced to these parameters the numerical analysis become more simplified while still retaining the same significance in results.
