\section{Nondimensionalization}
 
To help facilitate the analyses of this system, the full removal of all physical units is preferred. 
This process of nondimensionalization involves using known parameters to create substitutions with physical units cancelling.
Here the parameters used are: the biomass growth rate, $u$; the length of the region, $L$; and the maximum density for biomass and substrate, $M_\infty$ and $C_\infty$.
From using the following parameter changes, the system can be made unitless.
\begin{align}
  \chi &= \frac{x}{L} \implies L d\chi= dx \\
  \tau &= u t \implies \frac{1}{u} d\tau= dt \\
  \mathcal{M} &= \frac{M}{M_{\infty}} \\
  \mathcal{C} &= \frac{C}{C_{\infty}} \\
  \delta &= \frac{1}{u L^2} d \\
  \kappa &= \frac{k}{C_\infty} \\
  \nu &= \frac{n}{u C_\infty} \\
  \gamma &= \frac{M_\infty}{C_\infty} y
\end{align}

%%%%%%%% This section seems a little too pedantic for a thesis, Also it needs notations change and probably has errors....
% This gives the system
% \begin{align}
%   \mathcal{M}_\tau &= \frac{1}{u L^2} \nabla_\chi \left(D(m) \nabla_\chi M \right) + \frac{1}{u} F(\mathcal{C}) \mathcal{M} \\ 
%   \mathcal{C}_\tau &= \frac{ -1}{\mathcal{C}_\infty u} G(\mathcal{C}) \mathcal{M}
% \end{align}
% 
% where 
% \begin{equation}
%   D(M) = {u L^2 \delta} \frac{M^\alpha}{(1-M)^\beta}
% \end{equation}
% 
% \begin{equation}
%   F({\mathcal{C}}) = \frac{{u} {\mathcal{C} \mathcal{C}_0}}{{\kappa \mathcal{C}_0} + {\mathcal{C} \mathcal{C}_0}} M \left(1 - \left( \frac{M}{{\mathcal{C}}} \frac{M_0}{\mathcal{C}_0} \right)^\gamma \right) \\
% \end{equation}
% 
% \begin{equation}
%   G({\mathcal{C}}) = -\frac{{u \mathcal{C}_0 \nu} {\mathcal{C} \mathcal{C}_0}}{{\kappa \mathcal{C}_0} + {\mathcal{C} \mathcal{C}_0}} M \\
% \end{equation}
% 
% This can be greatly simplified by cancelling out parameters.
% 
% \begin{align}
%   M_\tau &= \nabla_\chi \left( {\delta} \frac{M^\alpha}{(1-M)^\beta} \nabla_\chi M\right) + \frac{ \mathcal{C} }{{\kappa } + {\mathcal{C}}} M \left(1 - \left( \frac{M}{{\mathcal{C}}} \frac{M_0}{\mathcal{C}_0} \right)^\gamma \right) \\
%   \mathcal{C}_\tau &= - \frac{\nu \mathcal{C}}{\kappa + \mathcal{C}} M
% \end{align}
% 
% Now we can name functions and get the final nondimensionalized system.

Using these, (\ref{equ:model_M}) and (\ref{equ:model_C}) can be simplified and nondimensionalized into, 
\begin{align} \label{equ:model_system}
  \mathcal{M}_\tau &= \nabla_\chi \left( D(\mathcal{M}) \nabla_\chi \mathcal{M} \right) + F(\mathcal{C}) \mathcal{M} \\
  \mathcal{C}_\tau &= - G(\mathcal{C}) \mathcal{M}, 
\end{align}

where,

\begin{equation}
\begin{aligned} \label{equ:model_functions}
  D(\mathcal{M}) &= \delta \frac{\mathcal{M}^\alpha}{(1 - \mathcal{M})^\beta} \\
  F(\mathcal{C}) &= \frac{ \mathcal{C}} {\kappa + \mathcal{C}} - \nu \\
  G(\mathcal{C}) &= \gamma \frac{\mathcal{C}}{\kappa +\mathcal{C}}.
\end{aligned}
\end{equation}
with only $\delta, \kappa, \nu, \gamma$ as model parameters. 

